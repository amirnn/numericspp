\documentclass[12pt]{article}
\usepackage[margin=1in]{geometry}
\usepackage{amsmath, amssymb}
\usepackage{hyperref}

\begin{document}

\title{18.335 / 18.4300 \\ Introduction to Numerical Methods}
\author{Lecture 1 Notes (Typed Version)}
\date{}
\maketitle

\section*{Course Overview}
\begin{itemize}
    \item Handouts, syllabus, psets, etc.
    \item Canvas + Piazza + Gradescope
    \item \url{https://github.com/mitmath/18335}
    \item \textit{Optional Julia tutorial this Friday (Feb. ??), 4--5:30pm @ 2-190}
    \item Parallel course this semester by Prof. Johnson: \textit{18.3?? Interdisciplinary Numerical Methods}
\end{itemize}

\noindent
\textbf{Numerical analysis} is:
\begin{itemize}
    \item Fast algorithms for approximating solutions to mathematical problems (with attention to accuracy).
    \item Tools for confirming that the math behind these algorithms works.
\end{itemize}

\section*{Example: Solve Nonlinear Equations}
We want to find $x^*$ such that $f(x^*) = 0$ for a function $f: \mathbb{R} \to \mathbb{R}$.

\subsubsection*{Bisection Method (Naive Approach)}
\begin{itemize}
    \item If $f(a)$ and $f(b)$ have opposite signs, then there is a root in $[a, b]$.
    \item Repeatedly bisect the interval until the remaining segment is sufficiently small.
\end{itemize}

\subsubsection*{Secant Method}
A faster alternative to bisection:
\begin{enumerate}
    \item Approximate $f(x)$ near $x_n$ by a straight line, using $f(x_n)$ and $f(x_{n-1})$.
    \item Solve that linear approximation $f(x) \approx 0$ for $x$ to get the next guess $x_{n+1}$.
\end{enumerate}
Formally,
\[
f(x) \approx f(x_n) + \frac{f(x_n) - f(x_{n-1})}{x_n - x_{n-1}} \, (x - x_n),
\]
which, upon setting $f(x) = 0$, yields the update
\[
x_{n+1} = x_n - f(x_n) \cdot \frac{x_n - x_{n-1}}{f(x_n) - f(x_{n-1})}.
\]
This often converges faster than bisection, typically with a rate related to the golden ratio ($\approx 1.618$).

\section*{Numerical Analysis Scope}
We deal with:
\begin{itemize}
    \item Nonlinear equations
    \item Data interpolation / approximation
    \item Linear algebra (solving $A x = b$, eigenvalue problems, etc.)
    \item Numerical differentiation and integration
    \item Partial Differential Equations (PDEs)
\end{itemize}

\subsubsection*{Examples}
\begin{itemize}
    \item \textbf{Poisson equation:}
    \[
    \Delta u = \frac{\partial^2 u}{\partial x^2} + \frac{\partial^2 u}{\partial y^2} = f,
    \quad u\big|_{\partial \Omega} = 0.
    \]
    Discretizing with finite differences leads to $A\mathbf{u} = \mathbf{b}$.

    \item \textbf{Schr\"{o}dinger equation:}
    \[
    -\Delta u + V\,u = E\,u,
    \]
    which, when discretized, becomes a large eigenvalue problem.
\end{itemize}

\section*{Outline}
\begin{itemize}
    \item Fundamentals of numerical linear algebra (condition numbers, stability, efficiency, etc.)
    \item Nonlinear optimization
    \item Numerical integration, FFT, and other topics
\end{itemize}

\end{document}
