\documentclass[12pt]{article}
\usepackage[margin=1in]{geometry}
\usepackage{amsmath,amssymb}
\usepackage{hyperref}

\begin{document}

\title{18.335 / 18.4300 \\ Introduction to Numerical Methods}
\author{Lecture 3 Notes (Typed Version)}
\date{}
\maketitle

\section*{Overview}
% Replace the text below with the actual Lecture 3 content after OCR / reading.
In this lecture, we continue discussing floating-point arithmetic and delve into more advanced aspects of numerical linear algebra. Topics may include:
\begin{itemize}
    \item More on backward stability
    \item Condition numbers in depth
    \item Special matrix decompositions (e.g., LU decomposition, Cholesky factorization)
\end{itemize}

\section*{Key Definitions and Concepts}
\begin{itemize}
    \item \textbf{Condition Number:} The ratio of relative output error to relative input error, often denoted $\kappa(A)$ for a matrix $A$.
    \item \textbf{Backward Error:} The minimal data perturbation that would make the computed solution exact for the perturbed problem.
    \item \textbf{Forward Error:} The difference between the computed solution and the true solution, measured in some norm.
\end{itemize}

\section*{LU Decomposition}
Suppose we have a matrix $A$. Under conditions (like nonsingularity of certain submatrices), $A$ can be factored as 
\[
A = L U,
\]
where $L$ is lower triangular (often with 1s on the diagonal) and $U$ is upper triangular.

\subsection*{Gaussian Elimination}
\begin{itemize}
    \item We systematically eliminate variables to convert $A$ into an upper-triangular form.
    \item The multipliers used in elimination populate $L$.
\end{itemize}

\noindent
\textbf{Partial Pivoting} is often used for numerical stability. 

\section*{Other Notes}
\begin{itemize}
    \item Additional details regarding pivoting strategies
    \item Rounding error considerations in matrix factorizations
    \item Case studies for well-conditioned vs. ill-conditioned systems
\end{itemize}

\section*{Summary}
This lecture emphasized:
\begin{itemize}
    \item The interplay between floating-point arithmetic and linear algebra
    \item How matrix decompositions relate to backward error analysis
    \item Why pivoting is crucial for stable Gaussian elimination
\end{itemize}

\end{document}