\documentclass[12pt]{article}
\usepackage[margin=1in]{geometry}
\usepackage{amsmath,amssymb}
\usepackage{hyperref}

\begin{document}

\title{18.335 / 18.4300 \\ Introduction to Numerical Methods}
\author{Lecture 3 Notes (Typed Version)}
\date{}
\maketitle

\section*{Overview}

This lecture continues the discussion of floating-point arithmetic, error analysis (especially catastrophic cancellation), and the concepts of backward and forward errors in numerical computations.

\section*{Catastrophic Cancellation}

\noindent
\textbf{Key Issue:} Subtracting two nearly equal numbers can cancel out the most significant digits, resulting in a large relative error.

\subsection*{Example Problem}

Evaluate $\displaystyle \frac{4}{4 - x} - \frac{4}{4 + x}$ for $x$ close to $0$.

\begin{itemize}
  \item \textbf{Method 1: Direct Evaluation.}
  \[
    S_1 = \mathrm{fl}\Bigl(\frac{4}{4 - x}\Bigr) - \mathrm{fl}\Bigl(\frac{4}{4 + x}\Bigr).
  \]
  Each division and subtraction can introduce floating-point errors $(1 + \delta_i)$. When $x$ is small, the two terms are nearly equal and result in a potentially large relative error.

  \item \textbf{Method 2: Re-arrange Calculation.}
  \[
    \frac{4}{4 - x} - \frac{4}{4 + x}
    = \frac{4(4 + x) - 4(4 - x)}{(4 - x)(4 + x)}
    = \frac{8x}{16 - x^2}.
  \]
  This form is often more accurate for $x$ near $0$, avoiding subtracting nearly equal quantities. This rearrangement leads to smaller relative error.
\end{itemize}

\section*{Another Example: Exponential and Logarithms}

Let us consider evaluating $\displaystyle e^{a+x} - e^a$. If $x$ is very small, a direct approach might lead to catastrophic cancellation.

\begin{itemize}
  \item \textbf{Method 1: Direct Evaluation.}
  \[
    \mathrm{Output}_1 = \mathrm{fl}\bigl(e^{a+x}\bigr) - \mathrm{fl}\bigl(e^a\bigr).
  \]
  Potentially large relative error if $e^{a+x} \approx e^a$.

  \item \textbf{Method 2: Factor and Re-arrange.}
  \[
    e^{a+x} - e^a = e^a \bigl(e^x - 1\bigr).
  \]
  Then compute $e^x - 1$ carefully using, e.g., series expansions or specialized library calls that reduce loss of significant digits when $x$ is small.  
\end{itemize}

\section*{Backward and Forward Error Review}

Recall from previous lectures:

\[
\text{Relative forward error} 
= \frac{\|\mathrm{fl}(x) - x_{\mathrm{exact}}\|}{\|x_{\mathrm{exact}}\|},
\]
\[
\text{Relative backward error}
= \min\Bigl\{ \epsilon : \mathrm{fl}(x) \text{ is the exact solution of } f(x+\Delta x) \text{ for some } \|\Delta x\| \le \epsilon \|x\|\Bigr\}.
\]

\subsection*{Stability Definitions}

\begin{itemize}
    \item \textbf{Backward stable:} If $\mathrm{fl}(x)$ is the \emph{exact} solution for a slightly perturbed problem. Formally, $\exists\,\Delta x \text{ with } \|\Delta x\| = O(\varepsilon_{\mathrm{mach}}) \|x\|\text{ such that } F(x+\Delta x) = F_{\mathrm{computed}}.$
    \item \textbf{Numerically stable:} If $\mathrm{fl}(x)$ is close in forward error to the true solution. Often implied by backward stability when the problem is well-conditioned.
\end{itemize}

Examples mentioned:
\begin{itemize}
    \item \textbf{Inner product} is backward stable under standard floating-point summation.
    \item \textbf{Outer product} is also backward stable (exercise).
\end{itemize}

\section*{Condition Number}

\noindent
The condition number $\kappa$ measures how input perturbations affect the output. For a function $f(\cdot)$ (differentiable), define
\[
\kappa = \sup_{\Delta x} \frac{\|Df(x)\,\Delta x\|}{\|f(x)\|} \cdot \frac{\|x\|}{\|\Delta x\|},
\]
or more simply (for linear problems $f(x)=Ax$):
\[
\kappa(A) = \|A\|\,\|A^{-1}\|.
\]
A high condition number indicates a more sensitive or ill-conditioned problem, where small changes in input can cause large changes in output.

\section*{Vector and Matrix Norms}

To measure errors, we use norms (particularly subordinate norms). For a vector $x\in \mathbb{R}^n$:

\begin{itemize}
    \item $\|x\|_1 = \sum_i |x_i|$,
    \item $\|x\|_2 = \sqrt{\sum_i |x_i|^2}$,
    \item $\|x\|_\infty = \max_i |x_i|$.
\end{itemize}

For matrices, a consistent (subordinate) norm $\|A\| = \sup_{x\neq 0} \frac{\|Ax\|}{\|x\|}.$

\subsection*{Examples}
\begin{itemize}
    \item $\|A\|_1 = \max_{j} \sum_i |a_{ij}|$, (max column sum).
    \item $\|A\|_\infty = \max_{i} \sum_j |a_{ij}|$, (max row sum).
    \item $\|A\|_2 = \sqrt{\lambda_{\max}(A^TA)}$, (spectral norm).
    \item $\|A\|_F = \sqrt{\sum_{i,j} |a_{ij}|^2}$, (Frobenius norm).
\end{itemize}

\section*{Combining Condition and Stability}

\subsection*{Forward Error Bound}
For a backward stable algorithm on a well-conditioned problem $f$, the forward error is typically:
\[
\|\mathrm{fl}(x) - x_{\mathrm{exact}}\| \;\le\; \kappa \,\times\, (\text{small backward error}).
\]

Hence, **backward error** plus **condition number** yields a **forward error** estimate.

\section*{Differentiable Functions \& Jacobians}

If $f:\mathbb{R}^n \to \mathbb{R}^n$ is differentiable, with Jacobian $Df(x)$, we approximate:
\[
f(x + \Delta x) \approx f(x) + Df(x)\,\Delta x.
\]
Thus, a small $\|\Delta x\|$ can yield a large $\|Df(x)\,\Delta x\|$ if $\|Df(x)\|$ is large, signifying ill-conditioning.

\section*{Summary}

\begin{itemize}
    \item Catastrophic cancellation arises when subtracting nearly equal numbers, which can lead to large relative errors.
    \item Rewriting expressions (e.g., factoring out common terms) helps mitigate significant digit loss.
    \item Backward stability: the computed result solves a slightly perturbed version of the original problem \emph{exactly}.
    \item Forward error relates to both backward error and the condition number of the underlying problem.
    \item Norms and condition numbers help us quantify how small input changes can affect the output.
\end{itemize}

\end{document}